%%%%%%%%%%%%%%%%%%%%%%%%%%%%%%%%%%%%%%%%%%%%%%%%%%%%%%%%%%%%%%%%%%%%%%%%
%%%%%%%%%%%%%%%%%%%%%% Simple LaTeX CV Template %%%%%%%%%%%%%%%%%%%%%%%%
%%%%%%%%%%%%%%%%%%%%%%%%%%%%%%%%%%%%%%%%%%%%%%%%%%%%%%%%%%%%%%%%%%%%%%%%

%%%%%%%%%%%%%%%%%%%%%%%%%%%%%%%%%%%%%%%%%%%%%%%%%%%%%%%%%%%%%%%%%%%%%%%%
%% NOTE: If you find that it says                                     %%
%%                                                                    %%
%%                           1 of ??                                  %%
%%                                                                    %%
%% at the bottom of your first page, this means that the AUX file     %%
%% was not available when you ran LaTeX on this source. Simply RERUN  %% 
%% LaTeX to get the ``??'' replaced with the number of the last page  %% 
%% of the document. The AUX file will be generated on the first run   %%
%% of LaTeX and used on the second run to fill in all of the          %%
%% references.                                                        %%
%%%%%%%%%%%%%%%%%%%%%%%%%%%%%%%%%%%%%%%%%%%%%%%%%%%%%%%%%%%%%%%%%%%%%%%%

%%%%%%%%%%%%%%%%%%%%%%%%%%%% Document Setup %%%%%%%%%%%%%%%%%%%%%%%%%%%%

% Don't like 10pt? Try 11pt or 12pt
\documentclass[12pt]{article}

% This is a helpful package that puts math inside length specifications
\usepackage{calc}

\usepackage{amsmath}
%\usepackage{enumitem}

% Layout: Puts the section titles on left side of page
\reversemarginpar

%
%         PAPER SIZE, PAGE NUMBER, AND DOCUMENT LAYOUT NOTES:
%
% The next \usepackage line changes the layout for CV style section
% headings as marginal notes. It also sets up the paper size as either
% letter or A4. By default, letter was used. If A4 paper is desired,
% comment out the letterpaper lines and uncomment the a4paper lines.
%
% As you can see, the margin widths and section title widths can be
% easily adjusted.
%
% ALSO: Notice that the includefoot option can be commented OUT in order
% to put the PAGE NUMBER *IN* the bottom margin. This will make the
% effective text area larger.
%
% IF YOU WISH TO REMOVE THE ``of LASTPAGE'' next to each page number,
% see the note about the +LP and -LP lines below. Comment out the +LP
% and uncomment the -LP.
%
% IF YOU WISH TO REMOVE PAGE NUMBERS, be sure that the includefoot line
% is uncommented and ALSO uncomment the \pagestyle{empty} a few lines
% below.
%

%% Use these lines for letter-sized paper
\usepackage[paper=letterpaper,
            %includefoot, % Uncomment to put page number above margin
            marginparwidth=1.2in,     % Length of section titles
            marginparsep=.05in,       % Space between titles and text
            margin=0.7in,               % 1 inch margins
            includemp]{geometry}

%% Use these lines for A4-sized paper
%\usepackage[paper=a4paper,
%            %includefoot, % Uncomment to put page number above margin
%            marginparwidth=30.5mm,    % Length of section titles
%            marginparsep=1.5mm,       % Space between titles and text
%            margin=25mm,              % 25mm margins
%            includemp]{geometry}

%% More layout: Get rid of indenting throughout entire document
\setlength{\parindent}{0in}

%% This gives us fun enumeration environments. compactenum will be nice.
\usepackage{paralist}

%% Reference the last page in the page number
%
% NOTE: comment the +LP line and uncomment the -LP line to have page
%       numbers without the ``of ##'' last page reference)
%
% NOTE: uncomment the \pagestyle{empty} line to get rid of all page
%       numbers (make sure includefoot is commented out above)
%
\usepackage{fancyhdr,lastpage}
\pagestyle{fancy}
%\pagestyle{empty}      % Uncomment this to get rid of page numbers
\fancyhf{}\renewcommand{\headrulewidth}{0pt}
\fancyfootoffset{\marginparsep+\marginparwidth}
\newlength{\footpageshift}
\setlength{\footpageshift}
          {0.5\textwidth+0.5\marginparsep+0.5\marginparwidth-2in}
\lfoot{\hspace{\footpageshift}%
       \parbox{4in}{\, \hfill %
                    \arabic{page} of \protect\pageref*{LastPage} % +LP
%                    \arabic{page}                               % -LP
                    \hfill \,}}

% Finally, give us PDF bookmarks
\usepackage{color,hyperref}
\definecolor{darkblue}{rgb}{0.0,0.0,0.3}
\hypersetup{colorlinks,breaklinks,
            linkcolor=darkblue,urlcolor=darkblue,
            anchorcolor=darkblue,citecolor=darkblue}

%%%%%%%%%%%%%%%%%%%%%%%% End Document Setup %%%%%%%%%%%%%%%%%%%%%%%%%%%%


%%%%%%%%%%%%%%%%%%%%%%%%%%% Helper Commands %%%%%%%%%%%%%%%%%%%%%%%%%%%%

% The title (name) with a horizontal rule under it
%
% Usage: \makeheading{name}
%
% Place at top of document. It should be the first thing.
\newcommand{\makeheading}[4]%
        {\hspace*{-\marginparsep minus \marginparwidth}%%
         \begin{minipage}[t]{\textwidth+\marginparwidth+\marginparsep}%
         \vspace{-0.6in}
         	\begin{center}
                {\large \bfseries #1}\\%[-0.15\baselineskip]%
                {\upshape #2}\\
                {\upshape #3}\\
                {\upshape #4}\\[-0.15\baselineskip]%
                 \rule{\columnwidth}{1pt}%
                 \end{center}
         \end{minipage}
         }

% The section headings
%
% Usage: \section{section name}
%
% Follow this section IMMEDIATELY with the first line of the section
% text. Do not put whitespace in between. That is, do this:
%
%       \section{My Information}
%       Here is my information.
%
% and NOT this:
%
%       \section{My Information}
%
%       Here is my information.
%
% Otherwise the top of the section header will not line up with the top
% of the section. Of course, using a single comment character (%) on
% empty lines allows for the function of the first example with the
% readability of the second example.
\renewcommand{\section}[2]%
        {\pagebreak[2]\vspace{1.3\baselineskip}%
         \phantomsection\addcontentsline{toc}{section}{#1}%
         \hspace{0in}%
         \marginpar{
         \raggedright \scshape #1}#2}

% An itemize-style list with lots of space between items
\newenvironment{outerlist}[1][\enskip\textbullet]%
        {\begin{enumerate}[#1]}{\end{enumerate}%
         \vspace{-.6\baselineskip}}

% An itemize-style list with little space between items
\newenvironment{innerlist}[1][\enskip\textbullet]%
        {\begin{compactenum}[#1]}{\end{compactenum}}

% To add some paragraph space between lines.
% This also tells LaTeX to preferably break a page on one of these gaps
% if there is a needed pagebreak nearby.
\newcommand{\blankline}{\quad\pagebreak[2]}

%%%%%%%%%%%%%%%%%%%%%%%% End Helper Commands %%%%%%%%%%%%%%%%%%%%%%%%%%%

%%%%%%%%%%%%%%%%%%%%%%%%% Begin CV Document %%%%%%%%%%%%%%%%%%%%%%%%%%%%

\begin{document}

\makeheading{Nicholas Penha Malaya}{318 Jarvis Hall}{Buffalo, NY 14260-4400}{nick@ices.utexas.edu}

%\section{Contact Information}
%
% NOTE: Mind where the & separators and \\ breaks are in the following
%       table.
%
% ALSO: \rcollength is the width of the right column of the table 
%       (adjust it to your liking; default is 1.85in).
%
%\newlength{\rcollength}\setlength{\rcollength}{1.85in}%
%%
%\begin{tabular}[t]{@{}p{\textwidth-\rcollength}p{\rcollength}}
%\href{http://www.ices.utexas.edu/}%
%     {Institute for Computational Engineering and Sciences} & \\
%\href{http://www.utexas.edu/}{The University of Texas at Austin}
%                           & \textit{Phone:} (512) 232-7791 \\
%           & \textit{Fax:} (512) 471-8694 \\
%201 E. 24th St.           & \textit{E-mail:}%
%\href{mailto:pbauman@ices.utexas.edu}{pbauman@ices.utexas.edu}\\
%Austin, TX 78712 USA    & \textit{}
%%\href{http://www.tedpavlic.com/}{www.tedpavlic.com}\\
%\end{tabular}

%\section{Security Clearance} 
%%
%Department of Defense Top Secret SCI with polygraph (expired: 2002) 

%==============================================================================
% NEW SECTION: Education
%==============================================================================
\section{Education}
%
\textbf{The University of Texas at Austin}, 
Austin, TX
\begin{itemize}

\item Ph.D., 
        Engineering, 2016
        \begin{itemize}
	 %\item \small{Numerical Simulation of Synthetic, 
	 %     Buoyancy-Induced Columnar Vortices}
        \item Advisor: Professor Robert D. Moser
        %\item Computational Engineering and Science option
        \end{itemize}

\item M.S., 
      Engineering, 2009 

\end{itemize}

\textbf{Georgetown University}, 
Washington, D.C. 
\begin{itemize}

\item B.S., 
        Physics \& Mathematics, \emph{with honors}, 2007
\end{itemize}
%        \begin{itemize}
%        \item Overall GPA: 3.57
%        \item Major GPA: 3.76
%        \end{itemize}


%==================================================================
% END OF FILE
%==================================================================


%==============================================================================
% NEW SECTION: Research Interests
%==============================================================================
\section{Research Interests}
%
Finite element methods for multi-physics and multi-scale problems with particular
emphasis on reacting fluid flow,
goal-oriented error estimation, adaptive modeling, high performance and
parallel computing,
coupling algorithms, stabilization methods, verification, validation,
and uncertainty quantification.

%==================================================================
% END OF FILE
%==================================================================



%==============================================================================
% NEW SECTION: Research Experience
%==============================================================================
\section{Research Experience}
%
\textbf{HPC Applications Software Technology}\\
\textbf{AMD Research} \hfill \textbf{Austin, TX}
%
\begin{itemize}
\item[] \textit{Computational Scientist} \hfill
	Dec. 2016 -- Present
\end{itemize}

\blankline


\textbf{Department of Mechanical Engineering}\\
\textbf{The University of Texas at Austin} \hfill \textbf{Austin, TX}
%
\begin{itemize}
\item[] \textit{Doctoral Candidate} \hfill
	Jan. 2014 -- Dec. 2016
\end{itemize}

\blankline

\textbf{Institute for Computational Engineering and Sciences}\\
\textbf{The University of Texas at Austin} \hfill \textbf{Austin, TX}
%
\begin{itemize}
\item[] \textit{Research Engineering/Scientist Associate II} \hfill
	Jan. 2010 -- Dec. 2013
\end{itemize}

\blankline

\textbf{Physics Laboratory, Optical Technology Division}\\
%, Optical Thermometry and Spectral Methods Group.
\textbf{National Institute of Standards and Technology} \hfill \textbf{Gaithersburg, MD}
%
\begin{itemize}
\item[] \textit{Assistant Researcher} \hfill
	May 2005 -- August 2005
\end{itemize}

%==============================================================================
% NEW SECTION: Awards
%==============================================================================
\section{Awards} 
%
\vspace{-0.3in}

\begin{itemize}
	\itemsep 0pt
	\item ``Best Paper" - 2012 TACC-Intel Highly Parallel Computing Symposium, April 10-11, Austin, TX
	\item Bruton Fellowship, 2006
	\item DOE Computational Science Graduate Fellowship, 2003 -- 2007
	\item CAM Graduate Fellowship, 2002
	\item Louis C. Wagner Scholarship, 2001
	\item Texas Offshore Industry Endowed Scholarship in ASE, 2000
	\item AP Scholar with honor, 1999
\end{itemize}

%==================================================================
% END OF FILE
%==================================================================


%==============================================================================
% NEW SECTION: Teaching Experience
%==============================================================================
\section{Teaching Experience}
%
.
%% \textbf{University at Buffalo, SUNY}
%% \begin{itemize}
%% \item[] \textbf{Graduate Level}
%%   \begin{itemize}
%%   \item[] \textit{MAE 610 - High Performance Computing II} \hfill \textbf{Spring 2014}
%%   \end{itemize}
%% \end{itemize}

%% \textbf{The University of Texas at Austin}
%% \begin{itemize}
%% \item[] \textbf{Graduate Level}
%%   \begin{itemize}
%%   \item[] \textit{EM/CAM 393N - Intro to Num. Methods for Fluids}%
%%     \hfill \textbf{Spring 2010}
%%     \begin{itemize}
%%     \item Co-taught with Roy H. Stogner, Benjamin S. Kirk, and Graham F. Carey
%%       %	\item Introductory graduate course introducing spatial and temporal discretization
%%       %              schemes for PDE's related to fluid mechanics
%%       %	\item Finite Difference, Finite Volume, and Finite Element methods
%%       %	\item Direct and iterative linear system solution strategies
%%       %	\item Nonlinear system solution methods
%%       %	\item Stabilization
%%       %	\item Verification
%%     \end{itemize}
%%   \end{itemize}
%% 	%
%% \item[] \textbf{Undergraduate Level}
%%   \begin{itemize}
%%   \item[] \textit{ASE 311 - Engineering Computations}%
%%     \hfill \textbf{Spring 2009}
%%     %        \begin{itemize}
%%     %        \item Undergraduate level
%%     %       \item Topics included: Floating point arithmetic, linear systems of equations, nonlinear equations and 
%%     %       nonlinear systems of equations, eigenvalues and eigenvectors, function approximation and interpolation,
%%     %       numerical integration and differentiation, numerical solutions of initial and boundary value problems
%%     %        \end{itemize}
%%     %
%%   \end{itemize}
%% \end{itemize}


%==============================================================================
% NEW SECTION: Refereed Publications
%==============================================================================
\section{Refereed Journal Publications}
%
%\begin{outerlist}
%\item [1.]

McMahan JA, Williams BJ, Smith RC, \textbf{Malaya N.}, A Linear Regression
Framework for the Verification of Bayesian Model Calibration
Algorithms. ASME. J. Verif. Valid. Uncert. 2017. doi:10.1115/1.4037705. 

\blankline

 Graham, J., Kanov, K., Yang X.I.A., Lee M.K., \textbf{Malaya, N.}, 
Lalescu, C.C., Burns, R., Eyink, G., Szalay, A., Moser, R.D. \& 
Meneveau, C. ``A Web Services-accessible database of turbulent channel
flow and its use for testing a new integral wall model for LES.''Journal
of Turbulence  (2015)

\blankline

M. Lee, \textbf{N. Malaya}, Rhys Ulerich, Robert D. Moser,  “Experiences from
Leadership Computing in Simulations of Turbulent Fluid Flows,”
Computing in Science \& Eng., vol. 16, no. 5, 2014, pp. 24–31.

\blankline

T. A. Oliver, \textbf{N. Malaya}, R. Ulerich, and R. D. Moser, “Estimating
uncertainties in statistics computed from direct numerical simulation,”
Phys. Fluids 26, 035101 (2014). http://dx.doi.org/10.1063/1.4866813 

\blankline

M. Lee, \textbf{N. Malaya}, and R. D. Moser, ``Petascale direct numerical sim-
ulation of turbulent channel flow on up to 786k cores,'' in Proceedings
of the 2013 ACM/IEEE International Conference for High Performance
Computing, Networking, Storage and Analysis (SC). ACM Press, 2013 {\bf(Best
Student Paper Finalist)}

\blankline

\textbf{N. Malaya}, K. C. Estacio-Hiroms, R. H. Stogner, K. W. Schulz, P. T. Bauman,
G. F. Carey, ``MASA: A Library for Verification Using Manufactured and
Analytical Solutions", Engineering with Computers, 29(4), 487--496, 2013

\blankline

\textbf{N. Malaya}, T. Oliver, K. C. Estacio-Hiroms, ``Manufactured Solutions for
the Favre-Averaged Navier-Stokes Equations 
with Eddy-Viscosity Turbulence Models'', Technical Paper, 50th AIAA ASM, 2012.

\blankline

\textbf{Nicholas Malaya}, Karl W. Schulz, Robert D. Moser
``Petascale I/O using HDF-5'', Teragrid'10 Technical Paper, Association for Computing Machinery,
August 2, 2010.

\blankline

Robert D. Moser, \textbf{Nicholas P. Malaya}, Henry Chang, et. al.,
``Theoretically based optimal large-eddy simulations'', Physics of Fluids, October 23, 2009.

%==================================================================
% END OF FILE
%==================================================================

%==============================================================================
% NEW SECTION: Accepted Publications
%==============================================================================
\section{Accepted Journal Publications}
E.~E.~Prudencio, \textbf{P.~T.~Bauman}, D.~Faghihi, K.~Ravi-Chandar, J.~T.~Oden,
``A computational framework for dynamic data-driven material damage control,
based on Bayesian inference and model selection",
Int. J. Numer. Meth. Engng, doi: 10.1002/nme.4669, 2014

%==================================================================
% END OF FILE
%==================================================================

%==============================================================================
% NEW SECTION: Submitted papers
%==============================================================================
\section{Submitted Publications}
%
T.~M.~van~Opstal, \textbf{P.~T.~Bauman}, S.~Prudhomme, and E.~H.~van~Brummelen,
``Goal-oriented model adaptivity for viscous, incompressible flow"

\blankline

E.~E.~Prudencio, \textbf{P.~T.~Bauman}, S.~V.~Williams, D.~Faghihi, K.~Ravi-Chandar,
J.~T.~Oden,
``Real-Time Inference of Stochastic Damage in Composite Materials"

%==================================================================
% END OF FILE
%==================================================================


%==============================================================================
% NEW SECTION: Book Chapters
%==============================================================================
\section{Book Chapters}
%
J. T. Oden, S. Prudhomme, \textbf{P. T. Bauman}, L. Chamoin,
``Estimation and Control of Modeling Error: A General Approach to Multiscale Modeling'',
in Multiscale Methods: Bridging the Scales in Science and Engineering, J. Fish (editor),
Oxford University Press, 2009.


%==============================================================================
% NEW SECTION: Technical Reports
%==============================================================================
%\section{Technical Reports}
%\textbf{P. T. Bauman}, J. T. Oden, E. E. Prudencio, S. Prudhomme, and K. Ravi-Chandar, ``Dynamic Data
%Driven Application Systems for Monitoring Damage in Composite Materials Under Dynamic Loads", ICES
%REPORT 12-37, The Institute for Computational Engineering and Sciences, The University of Texas at Austin,
%August 2012

%==============================================================================
% NEW SECTION: Papers in Preparation
%==============================================================================
\section{Papers in Preparation}
%
to do


%==============================================================================
% NEW SECTION: Refereed Conference Publications
%==============================================================================
\section{Refereed Conference Publications}
%
K. W. Schulz,  R. Ulerich,  N. Malaya, \textbf{P. T. Bauman}, R. Stogner, and Chris Simmons,
``Early Experiences Porting Scientific Applications to the Many Integrated Core (MIC) Platform",
TACC-Intel Highly Parallel Computing Symposium, Austin, TX,
April 10--11, 2012. Winner - Best Paper.

\blankline

\textbf{P. T. Bauman}, J. Jagodzinski, B. S. Kirk,
``Statistical Calibration of Thermocouple Gauges Used for Inferring Heat Flux", 
42nd AIAA Thermophysics Conference, Honolulu HI, June 27--30, 2011, AIAA-2011-3779.

\blankline

R. Stogner, \textbf{P. T. Bauman}, K. W. Schulz, and R. Upadhyay,
``Uncertainty and Parameter Sensitivity in Multiphysics Reentry Flows",
49th AIAA Aerospace Sciences Meeting including the New Horizons Forum and Aerospace Exposition,
Orlando, Florida, Jan. 4-7, 2011, AIAA Paper 2011-764.

\blankline

\textbf{P. T. Bauman}, R. Stogner, G. F. Carey, K. W. Schulz, R. Updadhyay, A. Maurente,
``Multiphysics Coupling for Reentry flows", 48th AIAA Aerospace Sciences Meeting,
Orlando, FL, Jan. 4--7, 2010, AIAA Paper 2010-1462.

\blankline

R.~R. Upadhyay and \textbf{P.~T.~Bauman} and R. Stogner and K.~W. Schulz and O.~A. Ezekoye, 
``Steady-State Ablation Model Coupling with Hypersonic Flow, 48th AIAA Aerospace Sciences Meeting,
Orlando, FL, Jan. 4--7, 2010, AIAA Paper 2010-1176.


%==============================================================================
% NEW SECTION: Invited Lectures
%==============================================================================
\section{Invited\\ Presentations}
%
%sandia?

\textbf{Nicholas Malaya}, ``Verification \& Software Quality in Scientific
      Computing'', Los Alamos Computational Physics Student Summer
      Workshop, 2014. 

\blankline

\textbf{Nicholas Malaya}, ``The Method of Manufactured Solutions'', Los Alamos
      Computational Physics Student Summer Workshop, 2013.

\blankline

\textbf{Nicholas Malaya}, Christopher Simmons, ``Scientific Software Engineering
      Best Practices'', Los Alamos Computational Physics Student Summer
      Workshop, 2012.

\blankline

\textbf{Nicholas Malaya} \& Robert D. Moser, ``Tools and Techniques for Code
Verification using Manufactured Solutions.'', SIAM Conference on
Uncertainty Quantification, April 2-4, 2012, Raleigh, North Carolina 

\blankline

\textbf{Nicholas Malaya}, Karl W. Schulz, ``Verification through the MASA
      Library'', Los Alamos Computational Physics Student Summer
      Workshop, 2011.

\blankline

\textbf{Nicholas Malaya}, Rhys Ulerich, Robert Moser, ``Petascale Direct
Numerical Simulations of Turbulent Channel Flow'', ESP Kick-off
Workshop and Project Plan Presentation, Oct. 18th, 2010.

\blankline

\textbf{Nicholas Malaya}``Theoretically Based Optimal LES'', TFS/NRE
Seminar, Department of Mechanical Engineering, Oct. 29th, 2009. 





%==============================================================================
% NEW SECTION: Conference Presentations
%==============================================================================
\section{Conference Presentations}
%
\textbf{N. Malaya}
Porting Scientific Applications using HIP
SC19 Booth Talk, Nov. 13th, 2018

\blankline

\textbf{N. Malaya}
AMD Exascale Applications and Software Technologies
PathForward Review Meeting, March 21st, 2018

\blankline

\textbf{N. Malaya}, R. Stogner, R. Moser,
Numerical Investigation of Synthetic Buoyancy-Induced Columnar Vortices,
Bulletin of the American Physical Society 60, 2015

\blankline

\textbf{Nicholas Malaya}
MASA: A Tool for the Verification of Scientific Software, SciPy20164
\blankline

\textbf{N. Malaya}, R. Ulerich, T. Oliver, R Moser, Estimating Uncertainties in
Statistics Computed from DNS, APS Meeting Abstracts 1, 21004, 2012 

\blankline

\textbf{Nicholas Malaya}, Karl Schulz,``Verification through the MASA Library'', DOE
PSAAP Annual Review, 2011

\blankline

\textbf{Nicholas Malaya},  Pk Yeung, ``Collaborative Research: Enabling Discovery in High Reynolds Number''
Turbulence via Advanced Tools for Petascale Simulation and Analysis, 2010

\blankline

\textbf{Nicholas Malaya}, Amitabh Bhattacharya, Robert Moser,
``Theoretically Based Optimal LES'', APS DFD, 2008.


%\pagebreak

%==============================================================================
% NEW SECTION: Technical Skills
%==============================================================================
\section{Technical Skills} 
%
Extensive experience developing numerical methods for simulation of physical phenomena
on workstation and parallel computing environments

\blankline

\textbf{Programming Experience}: 
\begin{itemize}
\itemsep 0pt
\item Languages: C/C++, Fortran, Python(Num/Scipy), \Latex{}, Bash, Mathematica, Octave/Matlab
\item Development Env: Linux, Emacs, Git, SVN, Buildbot, TravisCI
\item Libraries: MPI, OpenMP, HDF-5, FFTW, BLAS, Lapack, Hadoop, Dakota
\end{itemize}



%==============================================================================
% NEW SECTION: Other Skills
%==============================================================================
%\section{Other Experience}
%
%Outgoing personality and strong work ethic has proven to be effective in team-oriented projects.

%\begin{innerlist}
%\item Work effectively and efficiently in collaborative software development environment
%\item Clear, concise presentation style
%\end{innerlist}

%\pagebreak

%==============================================================================
% NEW SECTION: Service
%==============================================================================
\section{Service}
%
%% \textbf{Conference Organization}:
%% \begin{itemize}
%% \itemsep 0pt
%% \item Co-organizer for MAFELAP 2013 Minisymposium: ``Error Estimation and Adaptive Modeling''
%% \end{itemize}
%% \blankline

\textbf{Mentorship}:
\begin{itemize}
\itemsep 0pt
 \item Industry Mentor, RIPS IPAM UCLA, 2017, 2018
\end{itemize}

\blankline


\textbf{Committees}:
\begin{itemize}
\itemsep 0pt
 \item Program Committee Member, SciPy 2017, 2018
 \item Program Committee Member, PEARC17
 \item Program Committee, Great Lakes Consortium for Petascale Computation (2012--2015)
 \item Reviewer, Blue Waters Graduate Fellowship Program (2014--present)
\end{itemize}
\blankline


\textbf{Companies Advised}:
\begin{itemize}
\itemsep 0pt
 \item Board Member, kWh Analytics (2012-2016)
\end{itemize}

\blankline

\textbf{Journal Referee}:
\begin{itemize}
\itemsep 0pt
 \item Parallel Computing
 \item Journal of Fluid Mechanics
 \item Engineering with Computers
 \item Journal of Computational Physics
\end{itemize}


%==============================================================================
% NEW SECTION: Service
%==============================================================================
\section{Memberships}
%
SIAM, USACM, APS


%==============================================================================
% NEW SECTION: Applicable Coursework
%==============================================================================
%\section{Applicable Coursework} 
%%
%\textbf{Mathematics}:
%\begin{innerlist}
%\item Functional Analysis
%\item Methods of Applied Mathematics
%\item Theory of Probability
%\item Advanced Theory of Finite Elements
%\item Partial Differential Equations
%\end{innerlist}
%
%\blankline
%
%\textbf{Numerical Analysis/Computer Science}:
%\begin{innerlist}
%\item Numerical Linear Algebra
%\item Interpolation, Approximation, Quadrature, and Differential Equations
%\item Finite Element Methods
%\item Algorithms
%\end{innerlist}
%
%\blankline
%
%\textbf{Science/Engineering}:
%\begin{innerlist}
%\item Solid Mechanics
%\item Foundations of Fluid Mechanics
%\item Molecular Gas Dynamics
%\end{innerlist}

%==============================================================================
% NEW SECTION: Citizenship
%==============================================================================
\section{Citizenship}
%
United States of America

%==============================================================================
% NEW SECTION: Last Updated
%==============================================================================
\section{Last Updated}
\today


%\pagebreak
%==============================================================================
% NEW SECTION: References
%==============================================================================
\section{References}

% title and author
%\title{List of References}
%\author{Nicholas Malaya}

%-------------------------------------------------

\noindent Todd Oliver, Ph.d. \\
Research Scientist \\
Center for Predictive Engineering and Computational Science \\
Institute for Computational Engineering and Sciences \\
University of Texas at Austin \\
E-Mail: oliver@ices.utexas.edu \\
Phone: (617) 270-3124 \\
\newline
\newline
Karl W. Schulz, Ph.d. \\
Principal Engineer, Enterprise High Performance Computing Group, Intel \\
E-Mail: karl.w.schulz@intel.com \\
Phone: (512) 689-6913 \\
\newline
\newline
\noindent Professor Robert Moser \\
Institute for Computational Engineering and Sciences \\
and Department of Mechanical Engineering \\
University of Texas at Austin \\
E-Mail: rmoser@ices.utexas.edu \\
\noindent Phone: (512)-471-3168 


%%%%%%%%%%%%%%%%%%%%%%%%%% End CV Document %%%%%%%%%%%%%%%%%%%%%%
\end{document}
