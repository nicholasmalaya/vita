%%%%%%%%%%%%%%%%%%%%%%%%%%%%%%%%%%%%%%%%%%%%%%%%%%%%%%%%%%%%%%%%%%%%%%%%
%%%%%%%%%%%%%%%%%%%%%% Simple LaTeX CV Template %%%%%%%%%%%%%%%%%%%%%%%%
%%%%%%%%%%%%%%%%%%%%%%%%%%%%%%%%%%%%%%%%%%%%%%%%%%%%%%%%%%%%%%%%%%%%%%%%

%%%%%%%%%%%%%%%%%%%%%%%%%%%%%%%%%%%%%%%%%%%%%%%%%%%%%%%%%%%%%%%%%%%%%%%%
%% NOTE: If you find that it says                                     %%
%%                                                                    %%
%%                           1 of ??                                  %%
%%                                                                    %%
%% at the bottom of your first page, this means that the AUX file     %%
%% was not available when you ran LaTeX on this source. Simply RERUN  %% 
%% LaTeX to get the ``??'' replaced with the number of the last page  %% 
%% of the document. The AUX file will be generated on the first run   %%
%% of LaTeX and used on the second run to fill in all of the          %%
%% references.                                                        %%
%%%%%%%%%%%%%%%%%%%%%%%%%%%%%%%%%%%%%%%%%%%%%%%%%%%%%%%%%%%%%%%%%%%%%%%%

%%%%%%%%%%%%%%%%%%%%%%%%%%%% Document Setup %%%%%%%%%%%%%%%%%%%%%%%%%%%%

% Don't like 10pt? Try 11pt or 12pt
\documentclass[12pt]{article}

% This is a helpful package that puts math inside length specifications
\usepackage{calc}

\usepackage{amsmath}
%\usepackage{enumitem}

% Layout: Puts the section titles on left side of page
\reversemarginpar

%
%         PAPER SIZE, PAGE NUMBER, AND DOCUMENT LAYOUT NOTES:
%
% The next \usepackage line changes the layout for CV style section
% headings as marginal notes. It also sets up the paper size as either
% letter or A4. By default, letter was used. If A4 paper is desired,
% comment out the letterpaper lines and uncomment the a4paper lines.
%
% As you can see, the margin widths and section title widths can be
% easily adjusted.
%
% ALSO: Notice that the includefoot option can be commented OUT in order
% to put the PAGE NUMBER *IN* the bottom margin. This will make the
% effective text area larger.
%
% IF YOU WISH TO REMOVE THE ``of LASTPAGE'' next to each page number,
% see the note about the +LP and -LP lines below. Comment out the +LP
% and uncomment the -LP.
%
% IF YOU WISH TO REMOVE PAGE NUMBERS, be sure that the includefoot line
% is uncommented and ALSO uncomment the \pagestyle{empty} a few lines
% below.
%

%% Use these lines for letter-sized paper
\usepackage[paper=letterpaper,
            %includefoot, % Uncomment to put page number above margin
            marginparwidth=1.2in,     % Length of section titles
            marginparsep=.05in,       % Space between titles and text
            margin=0.7in,               % 1 inch margins
            includemp]{geometry}

%% Use these lines for A4-sized paper
%\usepackage[paper=a4paper,
%            %includefoot, % Uncomment to put page number above margin
%            marginparwidth=30.5mm,    % Length of section titles
%            marginparsep=1.5mm,       % Space between titles and text
%            margin=25mm,              % 25mm margins
%            includemp]{geometry}

%% More layout: Get rid of indenting throughout entire document
\setlength{\parindent}{0in}

%% This gives us fun enumeration environments. compactenum will be nice.
\usepackage{paralist}

%% Reference the last page in the page number
%
% NOTE: comment the +LP line and uncomment the -LP line to have page
%       numbers without the ``of ##'' last page reference)
%
% NOTE: uncomment the \pagestyle{empty} line to get rid of all page
%       numbers (make sure includefoot is commented out above)
%
\usepackage{fancyhdr,lastpage}
\pagestyle{fancy}
%\pagestyle{empty}      % Uncomment this to get rid of page numbers
\fancyhf{}\renewcommand{\headrulewidth}{0pt}
\fancyfootoffset{\marginparsep+\marginparwidth}
\newlength{\footpageshift}
\setlength{\footpageshift}
          {0.5\textwidth+0.5\marginparsep+0.5\marginparwidth-2in}
\lfoot{\hspace{\footpageshift}%
       \parbox{4in}{\, \hfill %
                    \arabic{page} of \protect\pageref*{LastPage} % +LP
%                    \arabic{page}                               % -LP
                    \hfill \,}}

% Finally, give us PDF bookmarks
\usepackage{color,hyperref}
\definecolor{darkblue}{rgb}{0.0,0.0,0.3}
\hypersetup{colorlinks,breaklinks,
            linkcolor=darkblue,urlcolor=darkblue,
            anchorcolor=darkblue,citecolor=darkblue}

%%%%%%%%%%%%%%%%%%%%%%%% End Document Setup %%%%%%%%%%%%%%%%%%%%%%%%%%%%


%%%%%%%%%%%%%%%%%%%%%%%%%%% Helper Commands %%%%%%%%%%%%%%%%%%%%%%%%%%%%

% The title (name) with a horizontal rule under it
%
% Usage: \makeheading{name}
%
% Place at top of document. It should be the first thing.
\newcommand{\makeheading}[4]%
        {\hspace*{-\marginparsep minus \marginparwidth}%%
         \begin{minipage}[t]{\textwidth+\marginparwidth+\marginparsep}%
         \vspace{-0.6in}
         	\begin{center}
                {\large \bfseries #1}\\%[-0.15\baselineskip]%
                {\upshape #2}\\
                {\upshape #3}\\
                {\upshape #4}\\[-0.15\baselineskip]%
                 \rule{\columnwidth}{1pt}%
                 \end{center}
         \end{minipage}
         }

% The section headings
%
% Usage: \section{section name}
%
% Follow this section IMMEDIATELY with the first line of the section
% text. Do not put whitespace in between. That is, do this:
%
%       \section{My Information}
%       Here is my information.
%
% and NOT this:
%
%       \section{My Information}
%
%       Here is my information.
%
% Otherwise the top of the section header will not line up with the top
% of the section. Of course, using a single comment character (%) on
% empty lines allows for the function of the first example with the
% readability of the second example.
\renewcommand{\section}[2]%
        {\pagebreak[2]\vspace{1.3\baselineskip}%
         \phantomsection\addcontentsline{toc}{section}{#1}%
         \hspace{0in}%
         \marginpar{
         \raggedright \scshape #1}#2}

% An itemize-style list with lots of space between items
\newenvironment{outerlist}[1][\enskip\textbullet]%
        {\begin{enumerate}[#1]}{\end{enumerate}%
         \vspace{-.6\baselineskip}}

% An itemize-style list with little space between items
\newenvironment{innerlist}[1][\enskip\textbullet]%
        {\begin{compactenum}[#1]}{\end{compactenum}}

% To add some paragraph space between lines.
% This also tells LaTeX to preferably break a page on one of these gaps
% if there is a needed pagebreak nearby.
\newcommand{\blankline}{\quad\pagebreak[2]}

%%%%%%%%%%%%%%%%%%%%%%%% End Helper Commands %%%%%%%%%%%%%%%%%%%%%%%%%%%

%%%%%%%%%%%%%%%%%%%%%%%%% Begin CV Document %%%%%%%%%%%%%%%%%%%%%%%%%%%%

\begin{document}

\makeheading{Paul Thomas Bauman}{201 E. 24th St.}{Austin, TX 78712}{ptbauman@gmail.com}

%\section{Contact Information}
%
% NOTE: Mind where the & separators and \\ breaks are in the following
%       table.
%
% ALSO: \rcollength is the width of the right column of the table 
%       (adjust it to your liking; default is 1.85in).
%
%\newlength{\rcollength}\setlength{\rcollength}{1.85in}%
%%
%\begin{tabular}[t]{@{}p{\textwidth-\rcollength}p{\rcollength}}
%\href{http://www.ices.utexas.edu/}%
%     {Institute for Computational Engineering and Sciences} & \\
%\href{http://www.utexas.edu/}{The University of Texas at Austin}
%                           & \textit{Phone:} (512) 232-7791 \\
%           & \textit{Fax:} (512) 471-8694 \\
%201 E. 24th St.           & \textit{E-mail:}%
%\href{mailto:pbauman@ices.utexas.edu}{pbauman@ices.utexas.edu}\\
%Austin, TX 78712 USA    & \textit{}
%%\href{http://www.tedpavlic.com/}{www.tedpavlic.com}\\
%\end{tabular}

%\section{Security Clearance} 
%%
%Department of Defense Top Secret SCI with polygraph (expired: 2002) 

%==============================================================================
% NEW SECTION: Education
%==============================================================================
\section{Education}
%
\textbf{The University of Texas at Austin}, 
Austin, TX
\begin{itemize}

\item Ph.D., 
        Engineering, 2016
        \begin{itemize}
	 %\item \small{Numerical Simulation of Synthetic, 
	 %     Buoyancy-Induced Columnar Vortices}
        \item Advisor: Professor Robert D. Moser
        %\item Computational Engineering and Science option
        \end{itemize}

\item M.S., 
      Engineering, 2009 

\end{itemize}

\textbf{Georgetown University}, 
Washington, D.C. 
\begin{itemize}

\item B.S., 
        Physics \& Mathematics, \emph{with honors}, 2007
\end{itemize}
%        \begin{itemize}
%        \item Overall GPA: 3.57
%        \item Major GPA: 3.76
%        \end{itemize}


%==================================================================
% END OF FILE
%==================================================================


%==============================================================================
% NEW SECTION: Research Interests
%==============================================================================
\section{Research Interests}
%
Finite element methods for multi-physics and multi-scale problems with particular
emphasis on reacting fluid flow,
goal-oriented error estimation, adaptive modeling, high performance and
parallel computing,
coupling algorithms, stabilization methods, verification, validation,
and uncertainty quantification.

%==================================================================
% END OF FILE
%==================================================================

%==============================================================================
% NEW SECTION: Awards
%==============================================================================
\section{Awards} 
%
\vspace{-0.3in}

\begin{itemize}
	\itemsep 0pt
	\item ``Best Paper" - 2012 TACC-Intel Highly Parallel Computing Symposium, April 10-11, Austin, TX
	\item Bruton Fellowship, 2006
	\item DOE Computational Science Graduate Fellowship, 2003 -- 2007
	\item CAM Graduate Fellowship, 2002
	\item Louis C. Wagner Scholarship, 2001
	\item Texas Offshore Industry Endowed Scholarship in ASE, 2000
	\item AP Scholar with honor, 1999
\end{itemize}

%==================================================================
% END OF FILE
%==================================================================

\pagebreak

%==============================================================================
% NEW SECTION: Research Experience
%==============================================================================
\section{Research Experience}
\textbf{Institute for Computational Engineering and Sciences}, Austin, TX
%
\begin{outerlist}
\item[] \textit{Research Associate} \hfill \textbf{Aug. 2010 -- Present}
\begin{innerlist}
\item Leading development of parallel, multi-physics, incompressible, reacting flow solver (GRINS)
\item Developing high-fidelity model of Nitridation experiment using low Mach number approximation to the 
Navier-Stokes equations
%	\begin{innerlist}
%	\item Develop low Mach number approximation to Navier-Stokes equations
%	\item Develop stabilization schemes
%	\item Implemented within GRINS solver
%	\item Final calibration to be done using Bayesian methods
%	\end{innerlist}
\item Co-Developer of fully implicit, finite element, hypersonic flow solver including surface ablation (FIN-S)
%	\begin{innerlist}
%	\item Formulated and implemented thermal non-equilibrium model with FIN-S
%	\item Formulated and implemented fully coupled, zero-dimensional surface ablation scheme
%	\end{innerlist}
\item Collaborated with Morel group at Texas A\&M on developing modeling error scheme for $S_N$ and $S_{PN}$
radiation transport models
\item Developed standalone application for Vacuum Arc Remelting simulation using GRINS
%	\begin{innerlist}
%	\item Implemented multiple physics models, including solidification model
%	\item Developed vector-valued element capability within libMesh library
%	\item Developed $H\left(\text{curl} \right)$ conforming element capabilities within libMesh library
%	\end{innerlist}
\item Continued investigation of statistical calibration of thermocouples for use as heat flux gauges
%	\begin{innerlist}
%	\item Developed multiple likelihood formulations to incorporate multiple types of data uncertainties
%	\end{innerlist}
\item Worked on discrete-velocity formulation of Boltzmann equation, including thread parallelism and testing
on Knights Ferry MIC Platform
%
%	\begin{innerlist}
%	\item Helped guide students with documentation and software development practices
%	\item Added thread parallelism with OpenMP and tested on new Knights Ferry MIC platform
%	\end{innerlist}
%
\item Collaborating on development of systems for monitoring damage in composite materials under dynamic loads
%
\item Supervising visiting Ph.D. student on developing adaptive modeling schemes for coupled Stokes and Navier-Stokes
equations
%
\end{innerlist}
\end{outerlist}

\blankline

\begin{outerlist}
\item[] \textit{Research Engineering/Scientist Associate III} \hfill \textbf{Jan. -- Aug. 2010}
\begin{innerlist}
\item Investigated statistical calibration of thermocouples for use as heat flux gauges
%	\begin{innerlist}
%	\item Scaling analysis to determine sensitive parameters
%	\item Processing of raw experimental data received from AEDC
%	\item Formulation of statistical inverse problem
%	\item Implementation of forward problem based on libMesh
%	\item Inverse problem implemented based on QUESO library
%	\end{innerlist}
\item Collaborated on development of chemistry dependent radiation model for PECOS full system simulation
\end{innerlist}
\end{outerlist}

\blankline

\begin{outerlist}
\item[] \textit{Post-Doctoral Fellow}%
        \hfill \textbf{May 2008 -- Jan. 2010}
\begin{innerlist}
\item Staff lead for coupled physics simulation
	%
	\begin{innerlist}
	\item Developed loose coupling formulation for hypersonic flow and in-house developed radiation
		and ablation models
	\item Worked with ablation and radiation modeling groups for software implementation
	\item NASA integrating developed software into their own codes
	\item Integration with uncertainty quantification software packages 
	\end{innerlist}
	%
\item Numerical simulation of chemically reacting, hypersonic flows using NASA codes
\item Investigating validation and uncertainty quantification methodologies within Bayesian framework as 
well as information theory
\item Implementation of new schemes within QUESO software package
	%
%	\begin{innerlist}
%	\item Implemented convergence metric within PECOS QUESO package
%	\item Investigating efficient implementation of Stochastic Newton algorithm
%	\end{innerlist}
	%
\item Maintained software repository for PECOS project
\item Development of parallel, C++ molecular statics code
\end{innerlist}
\end{outerlist}

\blankline

\begin{outerlist}
\item[] \textit{Graduate Research Assistant}%
        \hfill \textbf{Sept. 2002 -- May 2008}
\begin{innerlist}
\item Numerical simulation of polymeric materials
\item Theoretical development of numerical methods related to multiscale phenomena
\item Code development for molecular simulation, including parallel calculations using state-of-the-art
optimization algorithms
\item Code development for coupled particle/continuum simulation
\item Code development for \emph{a posteriori} error estimation and adaptivity within coupled
particle/continuum simulation
\end{innerlist}
\end{outerlist}

\blankline

\textbf{Institute for Advanced Technology}, Austin, TX
\begin{outerlist}
\item[] \textit{Senior Student Associate}% 
        \hfill \textbf{June 2000 -- Aug. 2002}
\begin{innerlist}
\item Performed numerical simulations of hypervelocity impact of projectiles
\item Simulation of formation of Odessa crater
\end{innerlist}

\end{outerlist}

%==============================================================================
% NEW SECTION: Teaching Experience
%==============================================================================
\section{Teaching Experience}
\textbf{The University of Texas at Austin}
\begin{outerlist}
\item[] \textit{EM/CAM 393N - Intro to Num. Methods for Fluids}%
	\hfill \textbf{Spring 2010}
	\begin{innerlist}
	\item Co-taught with Roy H. Stogner, Benjamin S. Kirk, and Graham F. Carey
	\item Introductory graduate course introducing spatial and temporal discretization schemes for PDE's related to fluid mechanics
	\item Finite Difference, Finite Volume, and Finite Element methods
	\item Direct and iterative linear system solution strategies
	\item Nonlinear system solution methods
	\item Stabilization
	\item Verification
	\end{innerlist}
	%
\item[] \textit{ASE 311 - Engineering Computations}%
        \hfill \textbf{Spring 2009}
        \begin{innerlist}
        \item Undergraduate numerical methods course
        \item Topics included: Floating point arithmetic, linear systems of equations, nonlinear equations and 
        nonlinear systems of equations, eigenvalues and eigenvectors, function approximation and interpolation,
        numerical integration and differentiation, numerical solutions of initial and boundary value problems
        \end{innerlist}
        %
\end{outerlist}

%==============================================================================
% NEW SECTION: Refereed Publications
%==============================================================================
\section{Refereed Journal Publications}
%
%\begin{outerlist}
%\item [1.]

McMahan JA, Williams BJ, Smith RC, \textbf{Malaya N.}, A Linear Regression
Framework for the Verification of Bayesian Model Calibration
Algorithms. ASME. J. Verif. Valid. Uncert. 2017. doi:10.1115/1.4037705. 

\blankline

 Graham, J., Kanov, K., Yang X.I.A., Lee M.K., \textbf{Malaya, N.}, 
Lalescu, C.C., Burns, R., Eyink, G., Szalay, A., Moser, R.D. \& 
Meneveau, C. ``A Web Services-accessible database of turbulent channel
flow and its use for testing a new integral wall model for LES.''Journal
of Turbulence  (2015)

\blankline

M. Lee, \textbf{N. Malaya}, Rhys Ulerich, Robert D. Moser,  “Experiences from
Leadership Computing in Simulations of Turbulent Fluid Flows,”
Computing in Science \& Eng., vol. 16, no. 5, 2014, pp. 24–31.

\blankline

T. A. Oliver, \textbf{N. Malaya}, R. Ulerich, and R. D. Moser, “Estimating
uncertainties in statistics computed from direct numerical simulation,”
Phys. Fluids 26, 035101 (2014). http://dx.doi.org/10.1063/1.4866813 

\blankline

M. Lee, \textbf{N. Malaya}, and R. D. Moser, ``Petascale direct numerical sim-
ulation of turbulent channel flow on up to 786k cores,'' in Proceedings
of the 2013 ACM/IEEE International Conference for High Performance
Computing, Networking, Storage and Analysis (SC). ACM Press, 2013 {\bf(Best
Student Paper Finalist)}

\blankline

\textbf{N. Malaya}, K. C. Estacio-Hiroms, R. H. Stogner, K. W. Schulz, P. T. Bauman,
G. F. Carey, ``MASA: A Library for Verification Using Manufactured and
Analytical Solutions", Engineering with Computers, 29(4), 487--496, 2013

\blankline

\textbf{N. Malaya}, T. Oliver, K. C. Estacio-Hiroms, ``Manufactured Solutions for
the Favre-Averaged Navier-Stokes Equations 
with Eddy-Viscosity Turbulence Models'', Technical Paper, 50th AIAA ASM, 2012.

\blankline

\textbf{Nicholas Malaya}, Karl W. Schulz, Robert D. Moser
``Petascale I/O using HDF-5'', Teragrid'10 Technical Paper, Association for Computing Machinery,
August 2, 2010.

\blankline

Robert D. Moser, \textbf{Nicholas P. Malaya}, Henry Chang, et. al.,
``Theoretically based optimal large-eddy simulations'', Physics of Fluids, October 23, 2009.

%==================================================================
% END OF FILE
%==================================================================

%==============================================================================
% NEW SECTION: Submitted papers
%==============================================================================
\section{Submitted Publications}
%
J.~T.~Oden, E.~E.~Prudencio, \textbf{P.~T.~Bauman}, ``Virtual Model Validation of Complex Multiscale Systems: 
Applications to Nonlinear Elastostatics", Comput. Methods Appl. Mech. Engrg., submitted.


%==============================================================================
% NEW SECTION: Book Chapters
%==============================================================================
\section{Book Chapters}
%
J. T. Oden, S. Prudhomme, \textbf{P. T. Bauman}, L. Chamoin, ``Estimation and Control of Modeling Error:
A General Approach to Multiscale Modeling'', in Multiscale Methods: Bridging the Scales in Science and Engineering, J. Fish (editor), Oxford University Press, 2009.

%==============================================================================
% NEW SECTION: Technical Reports
%==============================================================================
\section{Technical Reports}
\textbf{P. T. Bauman}, J. T. Oden, E. E. Prudencio, S. Prudhomme, and K. Ravi-Chandar, ``Dynamic Data
Driven Application Systems for Monitoring Damage in Composite Materials Under Dynamic Loads", ICES
REPORT 12-37, The Institute for Computational Engineering and Sciences, The University of Texas at Austin,
August 2012

%==============================================================================
% NEW SECTION: Papers in Preparation
%==============================================================================
\section{Papers in Preparation}

T.M. van Opstal, P.T. Bauman, S. Prudhomme, and E.H. van Brummelen, ``Goal-oriented model adaptivity for viscous, incompressible flow"

\blankline

M. Panesi, K. W. Schulz, P. T. Bauman, R. H. Stogner, C. S. Simmons, ``Incorporation of Chemistry Dependence into Radiation Models for Reentry Modeling"

\blankline

P. T. Bauman, T. A. Oliver, R. Moser, ``Statistical Calibration of Thermocouples for Inferring Heat Flux"

\blankline

B. S. Kirk, R. H. Stogner, T. A. Oliver, P. T. Bauman, ``Fully Implicit Navier-Stokes for Hypersonic Flow Including Surface Ablation"

\blankline

P. T. Bauman, R. Moser, ``Modeling of Vacuum Arc Remelting Using Two- and Three-Dimensional Finite Elements"

\blankline

P. T. Bauman, R. Moser, ``Statistical Calibration of Nitridation Reaction Parameters"

%==============================================================================
% NEW SECTION: Refereed Conference Publications
%==============================================================================
\section{Refereed Conference Publications}
%
K. W. Schulz,  R. Ulerich,  N. Malaya, \textbf{P. T. Bauman}, R. Stogner, and Chris Simmons, ``Early Experiences Porting Scientific Applications to the Many Integrated Core (MIC) Platform", TACC-Intel Highly Parallel Computing Symposium, Austin, TX,
April 10--11, 2012. Winner - Best Paper.

\blankline

\textbf{P. T. Bauman}, J. Jagodzinski, B. S. Kirk, ``Statistical Calibration of Thermocouple Gauges Used for Inferring Heat Flux", 42nd AIAA Thermophysics
Conference, Honolulu HI, June 27--30, 2011, AIAA-2011-3779.

\blankline

R. Stogner, \textbf{P. T. Bauman}, K. W. Schulz, and R. Upadhyay, ``Uncertainty and Parameter Sensitivity in Multiphysics Reentry Flows",
49th AIAA Aerospace Sciences Meeting including the New Horizons Forum and Aerospace Exposition, Orlando, Florida, 
Jan. 4-7, 2011, AIAA Paper 2011-764.

\blankline

\textbf{P. T. Bauman}, R. Stogner, G. F. Carey, K. W. Schulz, R. Updadhyay, A. Maurente,``Multiphysics Coupling for Reentry flows", 48th AIAA Aerospace Sciences Meeting, Orlando, FL,  Jan. 4--7, 2010, AIAA Paper 2010-1462.

\blankline

R.~R. Upadhyay and \textbf{P.~T.~Bauman} and R. Stogner and K.~W. Schulz and O.~A. Ezekoye, 
``Steady-State Ablation Model Coupling with Hypersonic Flow, 48th AIAA Aerospace Sciences Meeting, Orlando, FL,  
Jan. 4--7, 2010, AIAA Paper 2010-1176.

\blankline



%==============================================================================
% NEW SECTION: Invited Lectures%==============================================================================
\section{Invited\\ Presentations}
%
P. T. Bauman, ``Data Reduction Modeling in the Uncertainty Quantification Process", Simulation and Visualization Symposium, University
of Texas at San Antonio, San Antonio, TX, Nov. 12-13, 2012

\blankline

P. T. Bauman, S. Prudhomme, J. T. Oden, ``Goal-Oriented Error Estimation and Adaptive Multiscale Modeling Applied to the Simulation of Polymers used in Semiconductor Manufatcturing", Verification and Validation for Nuclear Systems Analysis Workshop II, Beach Cove Resort, North Myrtle Beach, SC, May 24-28, 2010

\blankline

P. T. Bauman, ``Adaptive Multiscale Modeling of Polymers for Semiconductor Manufacturing", Mechanics of Materials Summer Research Seminar, Department of Aerospace Engineering, The University of Texas at Austin, Aug. 8, 2008

\blankline

P. T. Bauman, ``Getting a Ph.D in Computational Science", Young Researcher Panel - Student Days Program, SC 2005, Seattle, WA, Nov. 12-18, 2005 


%==============================================================================
% NEW SECTION: Conference Presentations
%==============================================================================
\section{Conference Presentations}
%
P. T. Bauman, R. Moser, ``Data Reduction Modeling of a Graphite Nitridation Experiment", 
American Physical Society Division of Fluid Dynamics Meeting, San Diego, CA, Nov. 18--20, 2012.

\blankline

P. T. Bauman,  J. Jagodzinski, B. S. Kirk, ``Calibration of Thermocouple Gauges for Inferring Heat Flux through Statistical Inversion", USNCCM 11,
Minneapolis, MN, July 25--29, 2011.

\blankline

P. T. Bauman, J. Jagodzinski, B. S. Kirk, ``Statistical Calibration of Thermocouple Gauges Used for Inferring Heat Flux", 42nd AIAA Thermophysics
Conference, Honolulu HI, June 27--30, 2011.

\blankline

R. Moser, P. T. Bauman, ``Multiphysics Coupling Efforts within the PECOS Center", 3rd Annual AFOSR/NASA/SNL Ablation Workshop, Austin, TX, March 24--25, 2010.

\blankline

P. T. Bauman, R. Stogner, G. F. Carey, K. W. Schulz, R. Updadhyay, A. Maurente,``Multiphysics Coupling for Reentry flows", 48th AIAA Aerospace Sciences Meeting, Orlando, FL,  Jan. 4--7, 2010, AIAA Paper 2010-1462.

\blankline

P. T. Bauman, J. T. Oden, S. Prudhomme, ``Goal-Oriented Error Estimation and Adaptive Modeling of Coupled Particle and Continuum Models", MAFELAP 2009, Brunel University, Uxbridge, England, June 9--12, 2009.

\blankline

P. T. Bauman, S. Prudhomme, J. T. Oden, ``Adaptive Multiscale Modeling of Polymeric Materials 
used in Semiconductor Manufacturing", Nanotech 2009, Houston, TX, May 3--7, 2009.

\blankline

P. T. Bauman, J. Bass, J. T. Oden, S. Prudhomme, ``Adaptive Multiscale Modeling of Polymers with Arlequin Coupling'', USNCCM 9, San Francisco, CA, July 23 -- 26, 2007.

\blankline

P. T. Bauman, ``Adaptive Multiscale Modeling of Polymeric Materials'', Computational Science Graduate Fellowship Annual Fellows' Conference, Washington, D. C., June 19 -- 21, 2007.

\blankline

J. T. Oden, P. T. Bauman, S. Prudhomme, J. M. Bass,``Adaptive Multi-Scale Modeling: Goals Algorithms for Adaptive Error Control'', 17th US Army Symposium on Solid Mechanics, Baltimore, MD, April 2-5, 2007.

\blankline

P. T. Bauman, S. Prudhomme, J. T. Oden, ``Application of Goal-Oriented Adaptivity to Multiscale Models in Molecular Dynamics'', WCCM 7, Los Angeles, CA, July 16 -- 22, 2006.

\blankline

J. T. Oden, P. T. Bauman, S. Prudhomme, ``Estimation and Control of Modeling Error in Multi-Scale Simulation'', Conference on Adaptive Model Reduction in PDE Constrained Optimization, Rice University, Houston, TX, May 17 -- 19, 2006.

\blankline

P. Bauman, S. Prudhomme, and J. T. Oden, ``Application of Goal-Oriented Adaptive Modeling to Problems in Molecular Statics'', USNCCM 8, Austin, TX, July 25 -- 27, 2005.

\blankline 

S. Prudhomme, P. Bauman, and J. T. Oden, ``Reliability of Atomistic-Continuum Modeling Simulations for Problems in Molecular Statics'', ICCN 2005 Proceedings (Fifth International Conference on Computational Nanoscience and Nanotechnology), Anaheim, CA, May 8 -- 12, 2005.

\blankline

D. L. Littlefield, P. T. Bauman, ``Deformation of fluted rods subjected to severe lateral loads'', 13th U.S. Army TARDEC Ground Vehicle Survivability Symposium, Monterey, CA, April 9 - 11, 2002.

%\pagebreak

%==============================================================================
% NEW SECTION: Technical Skills
%==============================================================================
\section{Technical Skills} 
%
Extensive experience developing numerical methods for simulation of physical phenomena
on workstation and parallel computing environments

\blankline

\textbf{Programming Experience}: 
\begin{innerlist}
\item Developer: libMesh, QUESO
\item Proficient: C++, FORTRAN 90, PETSc, MPI, OpenMP, Autotools, Subverison, Environment Modules, Lmod
\item Functional: C, Emacs, VI, Python, TAO, Make, SCons, TBB
\item Minimal: Trilinos, DAKOTA
\end{innerlist}

\blankline

\textbf{Applications}: \LaTeX{}, B\textsc{ib}\TeX{}, ParaView, Matplotlib, Numpy, Octave, \textsc{Matlab}, Microsoft Office, 
Keynote, and other common productivity packages for Apple OS X, and Linux platforms
	
\blankline

\textbf{Operating Systems}: Linux, Apple OS X; minimal MS Windows XP/7


%==============================================================================
% NEW SECTION: Other Skills
%==============================================================================
\section{Other Experience}
%
Outgoing personality and strong work ethic has proven to be effective in team-oriented projects.

\begin{innerlist}
\item Work effectively and efficiently in collaborative software development environment
\item Clear, concise presentation style
\end{innerlist}

\pagebreak

%==============================================================================
% NEW SECTION: Service
%==============================================================================
\section{Service}
%
\textbf{Conference Organization}:
\begin{innerlist}
\item Co-organizer for MAFELAP 2013 Minisymposium: ``Error Estimation and Adaptive Modeling"
\item Co-organizer for SIAM CSE 2013 Minisymposium: ``Inverse Analysis and Uncertainty Quantification in Fluid Mechanics"
\item Co-organizer for MAFELAP 2009 Minisymposium: ``Goal-oriented Error Estimation and Adaptivity"
\end{innerlist}

\blankline

\textbf{Committees}:
\begin{innerlist}
\item DOE CSGF Screening Committee 2012--2014
\end{innerlist}

\blankline


\textbf{Journal Referee}:
\begin{innerlist}
\item Computer Methods in Applied Mechanics and Engineering
\item Computational Mechanics
\item AIAA Journal
\item AIAA Journal of Thermophysics and Heat Transfer
\end{innerlist}

%==============================================================================
% NEW SECTION: Applicable Coursework
%==============================================================================
%\section{Applicable Coursework} 
%%
%\textbf{Mathematics}:
%\begin{innerlist}
%\item Functional Analysis
%\item Methods of Applied Mathematics
%\item Theory of Probability
%\item Advanced Theory of Finite Elements
%\item Partial Differential Equations
%\end{innerlist}
%
%\blankline
%
%\textbf{Numerical Analysis/Computer Science}:
%\begin{innerlist}
%\item Numerical Linear Algebra
%\item Interpolation, Approximation, Quadrature, and Differential Equations
%\item Finite Element Methods
%\item Algorithms
%\end{innerlist}
%
%\blankline
%
%\textbf{Science/Engineering}:
%\begin{innerlist}
%\item Solid Mechanics
%\item Foundations of Fluid Mechanics
%\item Molecular Gas Dynamics
%\end{innerlist}

%==============================================================================
% NEW SECTION: Citizenship
%==============================================================================
\section{Citizenship}
%
United States

%\pagebreak
%==============================================================================
% NEW SECTION: References
%==============================================================================

%\section{References}
%%
%\textbf{Robert Moser}\\
%\textit{Professor}\\
%Department of Mechanical Engineering\\
%The University of Texas at Austin\\
%1 University Station C2200\\
%Austin, TX 78712\\
%(512) 471-0093\\
%rmoser@ices.utexas.edu
%
%
%\blankline
%
%\textbf{J. Tinsley Oden}\\
%\textit{Professor}\\
%Aerospace Engineering and Engineering Mechanics\\
%Institute for Computational Engineering and Sciences\\
%The University of Texas at Austin\\
%1 University Station C0200\\
%Austin, TX 78712-0027\\
%(512) 471-3312\\
%oden@ices.utexas.edu
%
%\blankline
%
%\textbf{Philip Varghese}\\
%\textit{Professor} \\
%Aerospace Engineering and Engineering Mechanics\\
%Institute for Computational Engineering and Sciences\\
%The University of Texas at Austin\\
%1 University Station C0200\\
%Austin, TX 78712-0027\\
%(512) 471-4596\\
%varghese@mail.utexas.edu
%
%\blankline
%
%\textbf{Serge Prudhomme}\\
%\textit{Research Scientist}\\
%Institute for Computational Engineering and Sciences\\
%The University of Texas at Austin\\
%1 University Station C0200\\
%Austin, TX 78712-0027\\
%(512) 475-8629\\
%serge@ices.utexas.edu
%
%%\blankline
%
%\pagebreak
%
%\textbf{C. Grant Willson}\\
%\textit{Professor}\\
%Department of Chemical Engineering\\
%The University of Texas at Austin\\
%1 University Station C0200\\
%Austin, TX 78712-0027\\
%(512) 471-7222\\
%willson@che.utexas.edu
%
%\blankline

%\textbf{Leszek Demkowicz}\\
%\textit{Professor}\\
%Aerospace Engineering and Engineering Mechanics\\
%The University of Texas at Austin\\
%%Institute for Computational Engineering and Sciences\\
%1 University Station C0200\\
%Austin, TX 78712-0027\\
%(512) 471-8694\\
%leszek@ices.utexas.edu

%\blankline

%\textbf{David L. Littlefield}\\
%\textit{Professor}\\
%Mechanical Engineering\\
%University of Alabama at Birmingham
%530 3rd Ave S, HOEN 330A\\
%Birmingham, AL 35294-4440\\
%(205) 975-5882\\
%dll@uab.edu

%%%%%%%%%%%%%%%%%%%%%%%%%% End CV Document %%%%%%%%%%%%%%%%%%%%%%
\end{document}
